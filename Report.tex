
\documentclass{article}
\usepackage{listings}
\usepackage{xcolor}
\usepackage{caption}
\usepackage{amsmath}

\captionsetup[lstlisting]{font={small,tt}}

\title{Documentation of Ludo Game Code}
\author{User}

\begin{document}

\maketitle

\section{Introduction}
This document provides an overview of the C code written for a Ludo game. The code is divided into several files, each responsible for different aspects of the game. The key functions and logic are described below.

\section{Function.c}
The \texttt{Function.c} file contains the core functions responsible for moving pieces, checking game conditions, and handling special scenarios such as energized or sick pieces.

\subsection{Key Functions}

\subsubsection{Movement and Positioning}
The code includes logic for moving pieces based on their state (e.g., energized or sick). If a piece is energized, it moves twice as far; if it is sick, its movement is halved.

\begin{lstlisting}[language=C,caption={Handling Movement in \texttt{Function.c}},label={lst:movement_function}]
if (previous_rolls[i] == 3 && dice[i] == 3 && mark== 1) {
    printf("[%s] rolled two consecutive 3s and %s piece move back kotuwa to base!\\n", player[i],color[i][temp2]);
    piece[i][temp2]=0; 
    position[i][temp2]=0;
}
previous_rolls[i] = dice[i];
\end{lstlisting}

\subsubsection{Special States Handling}
The file also includes logic for managing special states such as energized or sick pieces, influencing their movement over several rounds.

\begin{lstlisting}[language=C,caption={Handling Energized and Sick Pieces},label={lst:special_states}]
if (energized[i][j] == 1 && round_roll[i][j] <= 4) {
    dice[i] = dice[i] * 2;
    ...
}
if (sick[i][j] == 1 && round_roll[i][j] <= 4) {
    dice[i] = dice[i] / 2;
    ...
}
\end{lstlisting}

\section{main.c}
The \texttt{main.c} file is the entry point of the program. It initializes the game, handles user input, and coordinates the gameplay.

\subsection{Game Initialization}
This section initializes the game board and sets up the players.

\begin{lstlisting}[language=C,caption={Game Initialization in \texttt{main.c}},label={lst:initialization}]
int main() {
    initialize_board();
    ...
}
\end{lstlisting}

\section{behavior.c}
The \texttt{behavior.c} file contains the behavior logic for different players, potentially differentiated by their color (e.g., red, blue, yellow). Each player's strategy is defined in a specific function.

\subsection{Red Player Behavior}
The red player's behavior prioritizes moving pieces that are close to home.

\begin{lstlisting}[language=C,caption={Red Player Behavior},label={lst:red_behavior}]
int red_player_behavior(...) {
    int closest_to_home = -1;
    int closest_distance = 52;
    for (int i = 0; i < 4; i++) {
        if (piece[0][i] == 1) {
            int distance = (52 + position[0][i] - 0) % 52;  // 0 is red's home position
            if (distance < closest_distance) {
                closest_distance = distance;
                closest_to_home = i;
            }
        }
    }
    return closest_to_home != -1 ? closest_to_home : 0;
}
\end{lstlisting}

\subsection{Blue Player Behavior}
The blue player moves in a cyclic manner, ensuring all pieces are evenly moved.

\begin{lstlisting}[language=C,caption={Blue Player Behavior},label={lst:blue_behavior}]
int blue_player_behavior(...) {
    static int last_moved = -1;
    for (int i = 1; i <= 4; i++) {
        int current = (last_moved + i) % 4;
        if (piece[1][current] == 1) {
            last_moved = current;
            return current;
        }
    }
    return 0;
}
\end{lstlisting}

\end{document}
